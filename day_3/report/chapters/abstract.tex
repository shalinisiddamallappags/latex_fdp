\vspace{0.5cm}
\chapter*{\centering ABSTRACT}
	\thispagestyle{plain}
	\paragraph{} IoT, interconnection and communication between everyday objects, enables many application in many domains. The objective of this project is to create a model of IoT for a particular domain.
	
	\paragraph{} In this project, the sensors which are the primary source of information that detects the fault are interconnected which form a fault diagnosis system . The data aggregated from the sensor is processed by a microcontroller and later updated  to the server for global availability. The sensor network is connected to Raspberry Pi, that serves the purpose of a microcontroller and a server.
	
	\paragraph{} This project can be viewed as the integration of two major parts.The first as the fault detection system and the second for the management system which makes sure of the assistance for the detected fault.
	
	\paragraph{} The sensor detects the fault occurred and sends it to the Raspberry Pi in the form of code along with the vehicle current location traced by GPS  which is later updated to the server of the maintenance center through the Internet. In the second part ,the service man with the help of obtained data can remotely control the faulty part in the vehicle.
	
	\paragraph{} Implementing this project, we can overcome the problems in the vehicle at the earlier stage,and also get the assistance from the service station for the problem in the vehicle in comparatively very short span of time.
	
	\newpage
	